\documentclass[]{article}

\usepackage{color}
\usepackage{graphicx}
\usepackage{amsmath}
\usepackage{amsfonts}
\usepackage{mathrsfs}
\usepackage[symbol]{footmisc}
\newcommand{\red}[1]{\textbf{\color{red} #1}}

\usepackage[margin=1.5in]{geometry}

\setlength{\parindent}{0pt}
\setlength{\parskip}{0.5\baselineskip}

%opening
\title{Response to Reviewer 2}
\author{}
\date{}

\begin{document}

\maketitle

We are pleased with the positive assessment of our work by the reviewer and thank him/her for thoroughly reading our manuscript and for providing constructive input. The responses to the reviewer's comments and adaptations in the revised manuscript are discussed below.

\hrulefill

\paragraph{Reviewer} \textit{The paper presents very interesting research related to control of wind farms using yaw and induction control mechanism, as well as combination of both. Using large-eddy simulations (LES) combined with adjoint-based optimization the authors showed significant improvement in power generation for 4x4 wind farm layout. Both, uniform and turbulent inflow conditions were considered with 8 different control strategies for each case. The paper is well written and suggested for publication after addressing the following comments:}


\section*{Comments}
\hrulefill

\paragraph{1. Reviewer} \textit{The main novelty in current paper compared to Ref. [14] is the addition of the yawing. This should be clearly stated in the Introduction part, e.g. in the last paragraph (lines 107-118)}

\paragraph{Response} 

\hrulefill

\paragraph{2. Reviewer} \textit{Line 137: "...in an accurate model with full state...", references should be provided here with previous studies using this model to show its accuracy. Authors already cited them in Introduction but it would be worth to repeat some of them here.}

\paragraph{Response} 

\hrulefill

\paragraph{3. Reviewer} \textit{Line 184: L-BFGS-B. It might be good to spell it out for clarity.}

\paragraph{Response} 

\hrulefill

\paragraph{4. Reviewer} \textit{Line 202-203 should be placed before Eq. 13.}

\paragraph{Response} 

\hrulefill

\paragraph{5. Reviewer} \textit{Line 211: corresponding notation for dimensions should be provided, or reference to Table 1 should be given right after 4.8x2.4x1}

\paragraph{Response} 

\hrulefill

\paragraph{6. Reviewer} \textit{Line 217: Define Zh}

\paragraph{Response} 

\hrulefill

\paragraph{7. Reviewer} \textit{Line 215-218: Figure should be provided with 3D domain, to clarify the vertical placement of the turbine.}

\paragraph{Response} 

\hrulefill

\paragraph{8. Reviewer} \textit{It might be good to provide brief comments on sensitivity of the solution to control time or explain the choice of current control time. If it was done previously, reference would be enough.}

\paragraph{Response} 

\hrulefill

\paragraph{9. Reviewer} \textit{It looks like no towers were included in current simulation. The authors should clearly state it in Introduction and provide comments on importance/non-importance of tower inclusion in Summary part. It could be provided after the authors provided comments on the effect of stratification.}

\paragraph{Response} 

\hrulefill

\paragraph{10. Reviewer} \textit{In Table 1: the expression for precursor pressure gradient is inconsistent with the one defined in line 220, i.e. rho is missing.}

\paragraph{Response} 

\hrulefill

\paragraph{11. Reviewer} \textit{Line 335: The authors clearly stated that the optimal control is heterogeneous, however no comments on why it is heterogeneous were provided. It was stated that turbine can pick either wake meandering type of motion or wake redirection, however what will determine this decision?}

\paragraph{Response} 

\hrulefill

\paragraph{12. Reviewer} \textit{Figure 5: The blue color for C4 and green color for C2 very hard to see, especially on I3Y case.}

\paragraph{Response} 

\hrulefill

\paragraph{13. Reviewer} \textit{Figure 6 shows that for uniform inflow in a reference case the turbine in a second row will be producing almost 90\% less power then upstream. As a reader I would be interested to see previous studies on this. Adding corresponding references would be enough.}

\paragraph{Response} 

\hrulefill

\paragraph{14. Reviewer} \textit{Figure 7: Add tick labels to X-axis, i.e. to Time. (Same for Fig. 13)}

\paragraph{Response} 

\hrulefill

\paragraph{15. Reviewer} \textit{Line 424: "...steady uniform inflow cases", would it be better to use word "constant uniform inflow cases", to avoid any confusion with steady-state analysis.}

\paragraph{Response} 

\hrulefill

\paragraph{16. Reviewer} \textit{Line 596: the "and" is missing in "... yaw angle equation \_\_\_ the adjoint gradient"}

\paragraph{Response} 

\hrulefill

\paragraph{17. Reviewer} \textit{In Eq. A13 $rho_i$, $gamma_i$ and $chi_i$ should be defined. Currently they are defined in A.4.}

\paragraph{Response} 

\hrulefill

\end{document}
